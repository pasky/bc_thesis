\chapter{Pretty-printing kernel data structures}
Memory correctness plays the key role in any algorithm. Being able to swiftly
and efficiently examine arbitrarily complex data structures is a mandatory
trait of every modern toolchain. The GNU Debugger {\tt gdb}, Modular Debugger
{\tt mdb} and the LLVM Debugger {\tt lldb} fulfil the expectations with respect
to userland processes. Unfortunately, the kernel debugger DDB on FreeBSD (and
other systems using it) has been shipped without such functionality from
its inception many decades ago. We aim to solve this issue by creating a
mechanism to pretty-print data structures contained in the currently loaded
kernel image.

\section{Approach comparison}
Currently, DDB offers two techniques to access the memory which are affected by
many limitations, either lack of modularity or excessive straightforwardness.

\subsection{Low-level examination}
The {\tt examine} command [referencia!] in conjecture with its options such as
{\tt x} or {\tt f} is used to examine optional number of memory blocks. The
fact that the user must specify the encoding of the memory is a .

This feature of DDB is undoubtedly important in scenarios when a specific
bit or byte needs to be analysed, but falls short in examining complex
structures, as it literally forces user to use pen and paper to map the output
values to the data structure.

\subsection{Hard-coded structures}
The {\tt show} command [referencia!] improves upon the simplicity of th {\tt
examine} command by providing support for pretty-printing a restricted set of
data structures. Such structures are:
\begin{itemize}
	\item bio
	\item mutex
\end{itemize}

The problem with this approach lies in its {\it code change linearity}: every
time a data structures member changes, the corresponding code in DDB needs to
adapt. The same applies in situations when a new data structure is introduced -
a new pretty-printing code needs to be written.

\subsection{Universal CTF solution}
The logical extension of the {\tt show} command is to provide a way of encoding
type information during the compilation process and being able to read such
data later during the debugger runtime - a task that CTF is a great fit for -
so that there will be no need to alter the DDB source code in case of a
structure change in the kernel code-base. We therefore implemented such
approach by using the {\tt libctf} described in the Chapter 2. Following
sections describe and discuss the implementation details.

\section{Kernel-space adaptation}
By utilising the {\tt libctf} in the FreeBSDs kernel debugger, we place an
additional requirement on its availability. Due to security and usability
issues, {\tt libc} functions such as {\tt strlen} or {\tt malloc} are not
available in the kernel mode, but rather similar functions acting as their
replacement must be used. Notable change is in the {\tt malloc} function, which
expects two additional arguments. While its obligatory to have two builds of
the library - one using the {\tt libc} version of functions and one using the
kernel substitute - it is preferred to have an unified code-base. To achieve
this goal, a set of C pre-processor macros was created that are expanded to
different functions depending on the availability of the {\tt \_KERNEL} macro.

An example of the {\tt \_CTF\_STRDUP} implementation:
\begin{verbatim}
#ifdef _KERNEL
	#define _CTF_STRDUP(string) strdup(string, M_CTF)
#else
	#define _CTF_STRDUP(string) strdup(string)
#endif
\end{verbatim}

linker ctf get

\section{Type lookup}
\section{Pretty printing}
\section{Data structure speculation}
\section{Views}
\section{Corrupted structures}

