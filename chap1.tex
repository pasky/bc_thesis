\chapter{Compact C Type Format}

\section{Introduction}
The Compact C Type Format (CTF) is a debugging information format targeted on
the type system of the C99 [referencia!] language. It was conceived at Sun
Microsystems while working on the Modular Debugger and adopted by other
significant tools onwards. One central design goal of the CTF format is
contained in its name: compactness. The format utilises all kinds of
techniques, such as bit fields, ZIP compression [referencia!] or deduplication,
to achieve the most minimal form. This thesis discusses only the second version
of the format, since it is the only one used in production and contains more
up-to-date internals with respect to the modern C language specification.

\subsection{Contemporary usage}
Currently, the only two applications using the CTF data set are DTrace and the
Modular Debugger. The reason of the low adoption of the technology is caused by
insufficient documentation of the format, which this thesis aims to solve, and
multiple incoherent implementations with private unstable APIs and restrictive
licenses. This too is being solved as the part of this thesis.

\section{Format description}
Following sections will describe the CTF format and discuss the possible
improvement points that could be made in future versions of the format.

\subsection{ELF section}
As with all debugging information, the CTF data set is stored in the executable
file that it refers to. Since CTF aims to provide debugging information on all
Unix-like systems, it is designed to be using alongside the ELF binary format.
The data is stored in the section labeled \texttt{.SUNW\_ctf}.

\subsection{Header}
The CTF section begins with a preamble that indicates three information: 

\subsection{Types}
The type section is the heart of the format as it stores all type meta-data and
is referenced by other CTF sections - symbol types . It also contains other
type related information, such as type qualifiers, pointers, forward
declarations or padding structures.

Every type or a type related object contains a unique consecutive 16-bit ID
that is used as a reference. This poses a limitation of maximum 65536 types per
executable. Usually the type section is the largest of all CTF sections and
increasing the bit width of the ID property will enlarge the data set in a
significant way. On the other hand, as large single executables, such as the
kernel image or the Mozilla Firefox browser [referencia!], the number of types
used is steadily increasing.

[tabulka o tom ako rastie pocet typov na roznych kerneloch a ze kedy to
presiahne 65k]

Possible solution to the problem would be denoting the bit width of the ID
property in the CTF header or in a separate table section header. This in turn
would allow a great amount of elasticity across the spectrum of applications. 

[kolko base modulov ma menej nez 255 typov?]

Other meta-data stored about each type is the \texttt{kind} of the type, which
is a 4-bit integer. This poses a limitation of maximum 16 different
\texttt{kind}s of types. Currently, the second version of the CTF format has
assigned the first XXXXXXXXXXXX kinds with the remaining XXXXXXXXX reserved for
the future use. The mapping of the \texttt{kind} values can be found in the
\texttt{src/type/kind.h} file.

As the C language advances, it comprises of increasing number
of base types (C11 [referencia!] introduced XXXXXXXXXXXX new base types) and
CTF will have to adapt. The same suggested approach of configurable bit width
of the \texttt{kind} property might solve this issue and future-proof the
format.

Each following subsections describes one of the types or type related objects:

\subsubsection{Integer number}

\subsubsection{Floating point number}

\subsubsection{Enumeration}

\subsubsection{Structure and union}

\subsubsection{Function}

\subsubsection{New type definition}

\subsection{Function objects}

\subsection{Data objects}

\subsection{String table}
The string table ELF section \texttt{strtab} is considered to be a crucial
source of insight information into the binary as it stores names of all
symbols. CTF aims at the lowest disk usage possible and therefore the internal
string section that resides in the main CTF section is using indices into the
top-level ELF \texttt{strtab} in case there should be the same string stored in
both.  All internal data structures that store the name of any object do not
contain the actual data itself, but a mere reference into one of the string
tables (either the top-level ELF, or the internal CTF). Which table to use
is denoted as the first bit of the index number. It is not advised to remove or
alter the top-level ELF string table without editing the CTF data afterwards,
as it may (and probably will) introduce data errors.

\subsection{Labels and uniquification}

